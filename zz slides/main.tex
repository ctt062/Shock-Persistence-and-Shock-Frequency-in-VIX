% !TEX program = pdflatex
\documentclass[aspectratio=169,10pt]{beamer}

% ============================================================
%  Theme & Packages
% ============================================================
\usetheme{Madrid}
\usecolortheme{default}
\usepackage{booktabs}
\usepackage{amsmath,amssymb}
\usepackage{graphicx}
\usepackage{hyperref}
\usepackage{tikz}

% Path to figures folder (one level up)
\graphicspath{{../figures/}}

% Custom footer: show only institute (not author)
\setbeamertemplate{footline}{
  \leavevmode%
  \hbox{%
  \begin{beamercolorbox}[wd=.333333\paperwidth,ht=2.25ex,dp=1ex,center]{author in head/foot}%
    \usebeamerfont{author in head/foot}\insertshortinstitute
  \end{beamercolorbox}%
  \begin{beamercolorbox}[wd=.333333\paperwidth,ht=2.25ex,dp=1ex,center]{title in head/foot}%
    \usebeamerfont{title in head/foot}\insertshorttitle
  \end{beamercolorbox}%
  \begin{beamercolorbox}[wd=.333333\paperwidth,ht=2.25ex,dp=1ex,right]{date in head/foot}%
    \usebeamerfont{date in head/foot}\insertshortdate{}\hspace*{2em}
    \insertframenumber{} / \inserttotalframenumber\hspace*{2ex}
  \end{beamercolorbox}}%
  \vskip0pt%
}

% Smaller author font on title page
\setbeamerfont{author}{size=\small}

% ============================================================
%  Title
% ============================================================
\title[VIX Shock Persistence \& Frequency]{Shock Persistence and Shock Frequency in VIX}
\subtitle{A Quantitative Analysis of Volatility Dynamics}
\author{CHONG Tin Tak (20920359) \newline CHOI Man Hou (20894196) \newline Vittorio Prana CHANDREAN (20896895)}
\institute{HKUST - IEDA4000E}
\date{\today}

% ============================================================
\begin{document}

% ----------------------------------------------------------
\begin{frame}
\titlepage
\end{frame}

% ----------------------------------------------------------
\begin{frame}{Outline}
\tableofcontents
\end{frame}

% ============================================================
\section{Introduction}
% ============================================================

\begin{frame}{What is VIX?}
\begin{itemize}
    \item \textbf{VIX} = CBOE Volatility Index, derived from S\&P~500 option prices.
    \item Often called the ``fear gauge'' --- rises when markets expect turbulence.
    \item Understanding VIX dynamics is crucial for:
    \begin{itemize}
        \item Risk management and hedging
        \item Derivatives pricing
        \item Portfolio allocation
    \end{itemize}
\end{itemize}
\end{frame}

\begin{frame}{Research Questions}
\begin{enumerate}
    \item \textbf{How persistent is volatility?}\\
          How long does a VIX shock take to decay?
    \item \textbf{How frequently do large spikes occur?}\\
          Can we model extreme events as a point process?
    \item \textbf{Can we forecast VIX volatility?}\\
          Do GARCH-type models beat simple baselines out-of-sample?
\end{enumerate}
\end{frame}

% ============================================================
\section{Data}
% ============================================================

\begin{frame}{Data Overview}
\begin{itemize}
    \item \textbf{Source:} Yahoo Finance (ticker \texttt{\^{}VIX})
    \item \textbf{Period:} January 2010 -- November 2025
    \item \textbf{Observations:} 4,145 business days
    \item \textbf{Pre-processing:}
    \begin{itemize}
        \item Forward-fill missing dates
        \item 0.1\% winsorization to limit outlier influence
        \item Compute $\log(\text{VIX})$ and daily log-changes $\Delta\log(\text{VIX})$
    \end{itemize}
\end{itemize}
\end{frame}

\begin{frame}{VIX Time Series}
\begin{center}
    \includegraphics[width=0.85\textwidth]{vix_series.png}
\end{center}
\begin{itemize}
    \item Red markers indicate identified shock days (top 5\% of $\Delta\log$ VIX).
\end{itemize}
\end{frame}

% ============================================================
\section{Methodology}
% ============================================================

\begin{frame}{Volatility Modeling: GARCH vs.\ EGARCH}
\textbf{GARCH(1,1):}
\[
\sigma_t^2 = \omega + \alpha\,\varepsilon_{t-1}^2 + \beta\,\sigma_{t-1}^2
\]
\begin{itemize}
    \item Symmetric: positive and negative shocks have equal impact.
    \item Persistence $= \alpha + \beta$.
\end{itemize}

\vspace{0.5em}
\textbf{EGARCH(1,1):}
\[
\ln\sigma_t^2 = \omega + \beta\ln\sigma_{t-1}^2 + \alpha\bigl(|z_{t-1}| - \mathbb{E}|z|\bigr) + \gamma\, z_{t-1}
\]
\begin{itemize}
    \item Asymmetric via $\gamma$ (leverage effect).
    \item Models log-variance $\Rightarrow$ no positivity constraint.
\end{itemize}
\end{frame}

\begin{frame}{Shock Identification \& Arrival Process}
\begin{enumerate}
    \item Define a \textbf{shock} as $\Delta\log(\text{VIX}) \ge$ 95th percentile.
    \item Model inter-arrival times with:
    \begin{itemize}
        \item \textbf{HPP} (Homogeneous Poisson Process): constant rate $\lambda$.
        \item \textbf{NHPP} (Non-Homogeneous Poisson): time-varying $\lambda_t$ via Poisson GLM with lagged covariates.
    \end{itemize}
    \item Covariates are \emph{lagged} to avoid look-ahead bias (e.g., average $\log$ VIX from month $t{-}1$ predicts shocks in month $t$).
\end{enumerate}
\end{frame}

\begin{frame}{Forecast Evaluation Framework}
\begin{itemize}
    \item \textbf{Out-of-Sample Design:}
    \begin{itemize}
        \item Train on first 75\% of data.
        \item Monthly rolling re-estimation of GARCH.
        \item Forecast 1-step-ahead variance into the remaining 25\%.
    \end{itemize}
    \item \textbf{Baselines:}
    \begin{itemize}
        \item EWMA ($\lambda = 0.94$)
        \item 63-day rolling variance
    \end{itemize}
    \item \textbf{Metrics:}
    \begin{itemize}
        \item Log-score (predictive density evaluation)
        \item 95\% coverage rate
        \item PIT histogram (calibration diagnostic)
        \item Diebold--Mariano test (statistical significance)
    \end{itemize}
\end{itemize}
\end{frame}

% ============================================================
\section{Results}
% ============================================================

\begin{frame}{Volatility Model Comparison}
\begin{table}[ht]
\centering
\begin{tabular}{lcccc}
\toprule
Model & Distribution & AIC & Persistence & Half-life (days) \\
\midrule
GARCH(1,1) & GED & 27,507 & 0.852 & 4.3 \\
EGARCH(1,1) & GED & \textbf{27,372} & 0.934 & 10.8 \\
\bottomrule
\end{tabular}
\end{table}
\begin{itemize}
    \item EGARCH achieves lower AIC $\Rightarrow$ better fit.
    \item Higher persistence in EGARCH $\Rightarrow$ shocks decay more slowly ($\approx$11 days half-life).
    \item GED distribution selected automatically via PIT uniformity diagnostics.
\end{itemize}
\end{frame}

\begin{frame}{News Impact Curve (Asymmetry)}
\begin{center}
    \includegraphics[width=0.7\textwidth]{news_impact.png}
\end{center}
\begin{itemize}
    \item Positive shocks (VIX spikes) increase future variance more than negative shocks of equal magnitude decrease it.
\end{itemize}
\end{frame}

\begin{frame}{Q--Q Plot of Standardized Residuals}
\begin{center}
    \includegraphics[width=0.55\textwidth]{qq.png}
\end{center}
\begin{itemize}
    \item Points hug the 45° line in the tails $\Rightarrow$ GED captures fat tails well.
\end{itemize}
\end{frame}

\begin{frame}{ACF Comparison: Before \& After Filtering}
\begin{center}
    \includegraphics[width=0.75\textwidth]{acf.png}
\end{center}
\begin{itemize}
    \item Top: ACF of squared returns shows strong serial correlation (volatility clustering).
    \item Bottom: ACF of squared standardized residuals $\approx$ white noise $\Rightarrow$ GARCH successfully filters clustering.
\end{itemize}
\end{frame}

\begin{frame}{Shock Statistics}
\begin{itemize}
    \item \textbf{Threshold:} 95th percentile $= 0.1267$
    \item \textbf{Total shocks:} 208 events over 15 years
    \item \textbf{HPP rate:} $\approx 9$ shocks/year (95\% CI: 7.8--10.4)
    \item \textbf{NHPP:} Lagged $\log$ VIX coefficient $= -0.16$, indicating lower VIX levels predict fewer shocks next month.
\end{itemize}
\end{frame}

\begin{frame}{Monthly Shock Counts}
\begin{center}
    \includegraphics[width=0.85\textwidth]{shock_counts.png}
\end{center}
\begin{itemize}
    \item Notable clustering: 2011--12 (Euro crisis), 2018 (Volmageddon), 2020 (COVID), 2022 (rate hikes).
\end{itemize}
\end{frame}

\begin{frame}{Inter-Arrival Time Distribution}
\begin{center}
    \includegraphics[width=0.7\textwidth]{interarrival.png}
\end{center}
\begin{itemize}
    \item Histogram vs.\ exponential PDF: reasonable fit validates Poisson assumption for shock arrivals.
\end{itemize}
\end{frame}

\begin{frame}{Forecast Evaluation: Log-Scores}
\begin{table}[ht]
\centering
\begin{tabular}{lc}
\toprule
Model & Log-Score (higher = better) \\
\midrule
GARCH & 1.274 \\
EWMA & \textbf{1.375} \\
Rolling Var & 1.274 \\
\bottomrule
\end{tabular}
\end{table}
\begin{itemize}
    \item EWMA slightly outperforms GARCH out-of-sample.
    \item Diebold--Mariano $p \approx 0$ $\Rightarrow$ difference is statistically significant.
\end{itemize}
\end{frame}

\begin{frame}{Cumulative Log-Score Difference}
\begin{center}
    \includegraphics[width=0.8\textwidth]{cum_loss_diff.png}
\end{center}
\begin{itemize}
    \item Downward trend: EWMA consistently accumulates better scores.
    \item GARCH rarely catches up, even briefly.
\end{itemize}
\end{frame}

\begin{frame}{PIT Histogram (Calibration)}
\begin{center}
    \includegraphics[width=0.55\textwidth]{pit.png}
\end{center}
\begin{itemize}
    \item Near-uniform distribution indicates well-calibrated density forecasts.
    \item Mean $\approx 0.51$, std $\approx 0.26$ (ideal: 0.5, 0.29).
    \item 95\% coverage: 95.0\% (close to nominal).
\end{itemize}
\end{frame}

% ============================================================
\section{Discussion}
% ============================================================

\begin{frame}{Key Findings}
\begin{enumerate}
    \item \textbf{Persistence:} VIX volatility shocks have a half-life of 4--11 days depending on model; EGARCH captures longer memory.
    \item \textbf{Asymmetry:} Positive shocks increase variance more than negative shocks reduce it (leverage effect confirmed).
    \item \textbf{Shock Arrivals:} $\sim$9 large spikes per year; inter-arrival times follow exponential distribution, supporting Poisson modeling.
    \item \textbf{Forecasting:} Simple EWMA beats rolling GARCH on log-score, though calibration metrics are acceptable for both.
\end{enumerate}
\end{frame}

\begin{frame}{Limitations}
\begin{itemize}
    \item GARCH re-estimation is computationally expensive; fixed-parameter rolling used here.
    \item EWMA's superiority may reflect VIX's strong mean-reversion making elaborate models unnecessary.
    \item NHPP covariates limited to lagged VIX; macro/sentiment indicators could improve intensity modeling.
    \item Student-t/GED chosen automatically; skewed distributions not tested.
\end{itemize}
\end{frame}

\begin{frame}{Future Work}
\begin{itemize}
    \item Incorporate EGARCH into the OOS forecasting loop.
    \item Test realized volatility or high-frequency data for improved variance proxies.
    \item Extend NHPP with external regressors (VIX term structure, credit spreads).
    \item Deploy model as a real-time monitoring dashboard.
\end{itemize}
\end{frame}

% ============================================================
\section{Conclusion}
% ============================================================

\begin{frame}{Conclusion}
\begin{itemize}
    \item VIX exhibits \textbf{persistent, asymmetric volatility clustering} well captured by EGARCH.
    \item Large spikes arrive at $\sim$9/year and cluster around macro stress episodes.
    \item Out-of-sample, \textbf{EWMA remains a tough benchmark} to beat for density forecasting.
    \item The reproducible pipeline (\texttt{runall.py}) and diagnostic figures enable transparent, auditable research.
\end{itemize}

\vspace{1em}
\centering
\Large \textbf{Thank you!}\\[0.5em]
\normalsize Questions?
\end{frame}

% ============================================================
\section*{Appendix}
% ============================================================

\begin{frame}[fragile]{Appendix: Project Structure}
\small
\begin{verbatim}
Shock-Persistence-and-Shock-Frequency-in-VIX/
+-- runall.py
+-- src/
|   +-- config.py
|   +-- data_pipeline.py
|   +-- volatility_models.py
|   +-- shock_modeling.py
|   +-- forecast_evaluation.py
|   +-- visualization.py
+-- figures/
+-- notebooks/
+-- SUMMARY.md
+-- requirements.txt
\end{verbatim}
\end{frame}

\begin{frame}{Appendix: Key Equations}
\textbf{Log-Score:}
\[
S_t = \log f(r_t \mid \mu_t, \sigma_t^2)
\]

\textbf{PIT:}
\[
u_t = F(r_t \mid \mu_t, \sigma_t^2) \quad \text{should be } \mathrm{Uniform}(0,1)
\]

\textbf{Diebold--Mariano Statistic:}
\[
\text{DM} = \frac{\bar{d}}{\sqrt{\widehat{\mathrm{Var}}(\bar{d})}} \xrightarrow{d} N(0,1)
\]
where $d_t = L_t^{(A)} - L_t^{(B)}$ is the loss differential.
\end{frame}

\begin{frame}{Appendix: References}
\small
\begin{itemize}
    \item Bollerslev, T. (1986). Generalized autoregressive conditional heteroskedasticity. \textit{Journal of Econometrics}.
    \item Nelson, D. B. (1991). Conditional heteroskedasticity in asset returns: A new approach. \textit{Econometrica}.
    \item Diebold, F. X., \& Mariano, R. S. (1995). Comparing predictive accuracy. \textit{Journal of Business \& Economic Statistics}.
    \item Whaley, R. E. (2000). The investor fear gauge. \textit{Journal of Portfolio Management}.
\end{itemize}
\end{frame}

% ============================================================
\end{document}
